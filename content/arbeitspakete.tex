\chapter{Arbeitspakete}

\begin{table}[H]
    \tablestyle
    \tablealtcolored
    \begin{tabularx}{\textwidth}{Xllcr}
        \tableheadcolor
            \tablehead Name &
            \tablehead Kategorie &
            \tablehead Iteration &
            \tablehead Priorität &
            \tablehead Soll in Stunden
        \tabularnewline
        \tablebody
	    Latex Umgebung einrichten & Dokumentation & Inception 1 & 3     & 4 \tabularnewline
	    Latex Dokumentengerüst erstellen & Dokumentation & Inception 1 & 3     & 2 \tabularnewline
	    Dokumentvorlagen & Dokumentation & Inception 1 & 3     & 2 \tabularnewline
	    Visual Studio Online einrichten & Entwicklungsumgebung & Inception 1 & 2     & 2 \tabularnewline
	    Source Control einrichten & Entwicklungsumgebung & Inception 1 & 2     & 2 \tabularnewline
	    MSSQL Installieren & Implementation & Inception 1 & 2     & 2 \tabularnewline
	    Projektplan erstellen & Projektmanagement & Inception 1 & 1     & 8 \tabularnewline
	    Projektantrag & Projektmanagement & Inception 1 & 1     & 2 \tabularnewline
	    Iterationsplan & Projektmanagement & Inception 1 & 2     & 2 \tabularnewline
	    Risikomanagement & Projektmanagement & Inception 1 & 2     & 4 \tabularnewline
	    Projektsitzung & Projektmanagement & Inception 1 & 3     & 2 \tabularnewline
	    Domainmodell erstellen & Analyse & Elaboration 1 & 1     & 3 \tabularnewline
	    GUI Designs erstellen & Analyse & Elaboration 1 & 2     & 6 \tabularnewline
	    Domainanalyse & Analyse & Elaboration 1 & 2     & 2 \tabularnewline
	    Personas / Szenarios & Analyse & Elaboration 1 & 2     & 3 \tabularnewline
	    Operation Contract & Analyse & Elaboration 1 & 2     & 2 \tabularnewline
	    System State Diagram & Analyse & Elaboration 1 & 3     & 2 \tabularnewline
	    System Sequenz Diagram & Analyse & Elaboration 1 & 3     & 4 \tabularnewline
	    Activity Diagram & Analyse & Elaboration 1 & 2     & 4 \tabularnewline
    \tableend
    \end{tabularx}
    \caption{Arbeitspakete (1/3)}
\end{table}

\begin{table}[H]
    \tablestyle
    \tablealtcolored
    \begin{tabularx}{\textwidth}{Xllcr}
        \tableheadcolor
            \tablehead Name &
            \tablehead Kategorie &
            \tablehead Iteration &
            \tablehead Priorität &
            \tablehead Soll in Stunden
        \tabularnewline
        \tablebody
	    Logische Architektur & Design & Elaboration 1 & 1     & 4 \tabularnewline
	    Internes Design & Design & Elaboration 1 & 1     & 4 \tabularnewline
	    Prototypen & Implementation & Elaboration 1 & 2     & 8 \tabularnewline
	    Zeitplanung erstellen & Projektmanagement & Elaboration 1 & 3     & 2 \tabularnewline
	    Use Case in Brief Format & Requirements & Elaboration 1 & 2     & 4 \tabularnewline
	    Anforderungsspezifikationen & Requirements & Elaboration 1 & 2     & 4 \tabularnewline
	    Datenbankmodell erstellen &  Entwicklungsumgebung & Elaboration 1 & 1     & 4 \tabularnewline
	    Projektsitzung & Projektmanagement & Elaboration 1 & 3     & 2 \tabularnewline
	    Machbarkeitsanalyse & Projektmanagement & Elaboration 1 & 2     & 4 \tabularnewline
	    Externes Design & Analyse & Elaboration 2 & 2     & 6 \tabularnewline
	    Deployment einrichten & Entwicklungsumgebung & Elaboration 2 & 2     & 4 \tabularnewline
	    Testumgebung einrichten & Entwicklungsumgebung & Elaboration 2 & 2     & 4 \tabularnewline
	    Automatisches Testing & Entwicklungsumgebung & Elaboration 2 & 2     & 4 \tabularnewline
	    Automatisches Building & Entwicklungsumgebung & Elaboration 2 & 2     & 4 \tabularnewline
	    Visual Studio für Deployment konfigurieren & Entwicklungsumgebung & Elaboration 2 & 1     & 4 \tabularnewline
	    Einrichten Infrastruktur & Implementation & Elaboration 2 & 1     & 8 \tabularnewline
	    IIS Konfigurieren & Implementation & Elaboration 2 & 1     & 2 \tabularnewline
	    Use Case in Fully Dressed Format & Requirements & Elaboration 2 & 3     & 6 \tabularnewline
	    Projektsitzung & Projektmanagement & Elaboration 2 & 3     & 2 \tabularnewline
	    Loginfunktion & Implementation & Construction 1 & 2     & 4 \tabularnewline
    \tableend
    \end{tabularx}
    \caption{Arbeitspakete (2/3)}
\end{table}

\begin{table}[H]
    \tablestyle
    \tablealtcolored
    \begin{tabularx}{\textwidth}{Xllcr}
        \tableheadcolor
            \tablehead Name &
            \tablehead Kategorie &
            \tablehead Iteration &
            \tablehead Priorität &
            \tablehead Soll in Stunden
        \tabularnewline
        \tablebody
	    Logoutfunktion & Implementation & Construction 1 & 2     & 2 \tabularnewline
	    Events anzeigen & Implementation & Construction 1 & 1     & 4 \tabularnewline
	    Eventdetails anzeigen & Implementation & Construction 1 & 2     & 2 \tabularnewline
	    Benutzerverwaltung CRUD & Implementation & Construction 1 & 1     & 6 \tabularnewline
	    Logging & Implementation & Construction 1 & 4     & 3 \tabularnewline
	    Bugfixing & Implementation & Construction 1 & 2     & 3 \tabularnewline
	    Code-Review & Qualitätsmanagement & Construction 1 & 4     & 2 \tabularnewline
	    Usability Tests & Qualitätsmanagement & Construction 1 & 4     & 2 \tabularnewline
	    Projektsitzung & Projektmanagement & Construction 1 & 3     & 2 \tabularnewline
	    Helfereinsätze für Event anzeigen & Implementation & Construction 2 & 2     & 4 \tabularnewline
	    Helfereinsatz anmelden & Implementation & Construction 2 & 1     & 2 \tabularnewline
	    Helfereinsatz abmelden & Implementation & Construction 2 & 2     & 2 \tabularnewline
	    Helfereinsatzdetails anzeigen & Implementation & Construction 2 & 3     & 4 \tabularnewline
	    Einsätze von Mitgliedern anzeigen & Implementation & Construction 2 & 3     & 4 \tabularnewline
	    Events verwalten CRUD & Implementation & Construction 2 & 1     & 6 \tabularnewline
	    Helfereinsätze verwalten CRUD & Implementation & Construction 2 & 1     & 4 \tabularnewline
	    Helfereinsätze zuordnen & Implementation & Construction 2 & 2     & 4 \tabularnewline
	    Bugfixing & Implementation & Construction 2 & 2     & 3 \tabularnewline
	    Code-Review & Qualitätsmanagement & Construction 2 & 4     & 2 \tabularnewline
	    Usability Tests & Qualitätsmanagement & Construction 2 & 4     & 2 \tabularnewline
	    Projektsitzung & Projektmanagement & Construction 2 & 3     & 2 \tabularnewline
	    Events importieren (Verbands API) & Implementation & Construction 3 & 3     & 8 \tabularnewline
	    Mitglieder importieren (webling API) & Implementation & Construction 3 & 3     & 8 \tabularnewline
	    Bugfixing & Implementation & Construction 3 & 2     & 3 \tabularnewline
	    Code-Review & Qualitätsmanagement & Construction 3 & 4     & 2 \tabularnewline
	    Usability Tests & Qualitätsmanagement & Construction 3 & 4     & 2 \tabularnewline
	    Projektsitzung & Projektmanagement & Construction 3 & 3     & 2 \tabularnewline
    \tableend
    \end{tabularx}
    \caption{Arbeitspakete (3/3)}
\end{table}