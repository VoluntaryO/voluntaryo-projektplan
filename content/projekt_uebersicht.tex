\chapter{Projekt Übersicht}
Für einen Unihockey Verein (namentlich FB Riders) soll der Spieltrieb während der Saison vereinfacht werden. Ein grosser Aufwand für die Planung der Saison ist die Einteilung der Helfereinsätze, welche es für jede Spielrunde braucht, über den gesamten Verein (ca. 250 Mitglieder) zu organisieren.
\\

Für die Verteilung aller Einsätze wird von einem Kontingent (Anzahl Helferstunden) gesprochen, welches für jede Mannschaft bestimmt wird. Ein Spieltag wird in einzelne Einsätze unterteilt und schlussendlich werden dem Spieltag eine oder mehrere Mannschaften für die Erfüllung der Einsätze zugeteilt. Die Mannschaft und deren Spieler sind nachher in der Pflicht, dieses Kontingent zu erreichen und alle zugeteilten Einsätze abzudecken.
\\
Natürlich sollen die Helfer ihre Einsätze bestmöglich nach ihren Präferenzen wählen können. Daher sollen die Einsätze für eine Saison von einer zentralen Stelle erfasst und später von den Mitgliedern eingetragen werden.
\\
Damit auch immer alle Helfer korrekt erscheinen und informiert sind, soll für den Termin jeweils eine Erinnerung verschickt werden. 
\\
Die Helfereinsätze stellen zum Teil spezielle Anforderungen an die helfende Person (bspw. mind. 18 Jahre), wodurch die Suche weiter erschwert werden kann. 
\\
Anbindung an die Mitgliederdatenbank auf \href{http://www.webling.ch/}{webling.ch} ist denkbar. 

\section{Zweck und Ziel}
In einem realen Umfeld soll ein Softwareengineering-Projekt mit Hilfsmitteln, die heute oder in naher Zukunft verwendet werden, umgesetzt werden. 
\\
Das Ziel ist eine funktionierende Webplattform, welche die Abwicklung der Helfereinsätze zu einem Teil automatisiert und grundlegend vereinfacht. Dem Hauptbenutzer soll eine ansprechende und konsistente Oberfläche geboten werden.

\subsection{Persönliche Ziele}
Wir wollen in dem Softwareengineering-Projekt eine produktive Software mit dem Erlernten aus den Kursen SE1 und SE2 umsetzen. Es sollen essentielle Techniken aus dem Softwareengineering angewendet und Erfahrungen gesammelt werden. Die Möglichkeiten im Microsoftumfeld interessieren uns sehr und wir möchten daher die Umsetzung mit Microsoft Technologien angehen.

\section{Lieferumfang}
\subsection{Grundfunktionen}
\begin{itemize}
    \item Mitgliederanmeldung (Import Mitglieder von webling.ch Service, Login / Logout, Teamzugehörigkeit)
    \item Benutzersystem mit Rollen (Mitglied, Planer, Admin)
    \item Einsatzplanung (inkl. Anbindung Verbandsdaten)
    \item Helfereinsatzvergabe nach Mannschaft
    \item Eintragen/Anmelden von Mitglied für Helfereinsatz
    \item Ansicht der eingetragenen Helfer / Ansicht der Einsätze eines Mitglieds
    \item Exportfunktion der Termine (Kalender, iCal)
    \item Erinnerung der Helfer vor Einsatz via E-Mail
    \item Variante für Festlegung Helfereinsätze Stunden / Mitglied
    \item Variante für Festlegung Helfereinsätze Studen  / Mannschaft
\end{itemize}

\section{Annahmen und Einschränkungen}
Es werden sich ca. 250 Mitglieder anmelden können.
Externe Webservices und Datenbestände können verwendet werden.
\\ Eine Abnahme während des Projekts durch den Verein ist nicht geplant!