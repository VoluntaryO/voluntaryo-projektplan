\chapter{Qualitätsmassnahmen}

\section{Dokumentation}
Die Dokumentation wird in Latex erstellt und mit GIT als Versionskontrollsystem auf \href{https:www.//github.com}{GitHub} verwaltet. Dies bietet uns einige Vorteile zur Sicherstellung der Qualität der Dokumentation:
\\\begin{itemize}
    \item Gemeinsames bearbeiten mit Konfliktlösung (GIT)
    \item Automatische Versionierung des Dokumentes und Übersicht der Änderungen
    \item Eine zentrale Ablage für das Dokument
\end{itemize}


\section{Projektmanagement}
Als Projektmanagement Tool kommt \href{https://www.atlassian.com/software/jira/agile}{Atlassian JIRA} zum Einsatz. Sämtliche Ressourcen werden damit 
tracked und gemanaged. Die Arbeitspakete werden in JIRA erfassst, priorisiert, den Teammitgliedern zugeteilt und dann implementiert. 
Die JIRA-Instanz ist erreichbar über diese URL: 
\\\url{http://sinv-56086.edu.hsr.ch:40010}

\section{Entwicklung}
Der gesamte Quellcode des Projekts wird mithilfe des Team Foundation Servers verwaltet und versioniert. Wir setzen dabei auf das Cloud-Angebot von \href{http://www.visualstudio.com/}{Visual Studio Online} 
\\Es darf nur Code veröffentlicht werden, der einen sinnvollen Zwischenstand darstellt und der lauffähig und getestet ist.
\subsection{Vorgehen}

\subsection{Unit Testing}
Während der Entwicklung sollten für alle Funktionen, bei denen es sinnvoll ist, Unit-Tests durchgeführt. Bevor der Quellcode eingecheckt werden kann müssen diese Tests erfolgreich sein.
\\Das Bereitstellen eines möglichen Unit-Tests ist integraler Bestandteil eines Arbeitspaketes.
\begin{table}[H]
    \tablestyle
    \tablealtcolored
    \begin{tabularx}{\textwidth}{l X l}
        \tablebody
        \textbf{Unit Testing} &
            Während der Entwicklung sollten für alle Funktionen, bei denen es sinnvoll ist, Unit-Tests durchgeführt. Bevor der Quellcode eingecheckt werden kann müssen diese Tests erfolgreich sein.
            \tabularnewline
        \textbf{System Testing} &
            Nach jeder abgeschlossenen Iteration wird der aktuelle Build ausgiebig getestet. Daraus resultiert ein Testbericht. Aus den resultierenden Fehlern werden Incidents eröffnet.
            \tabularnewline
        \textbf{Usability Testing} &
            Um kontinuierlich externes Feedback zu erhalten und um die Bedienbarkeit zu testen, wird das Projekt nach jeder Iteration an ausgewählte Benutzer verteilt, um ihnen die Möglichkeit für Feedback zu geben. So können wir sicherstellen, dass das Endprodukt intuitiv bedient werden kann.
            \tabularnewline
        \tableend
    \end{tabularx}
    \caption{Unit Testing}
\end{table}

\subsection{Code Reviews}
Um die Motivation zur Erstellung von wartbarem Code und den Austausch von Programmiertechniken zu fördern, werden wir unseren Code gegenseitig Reviewen. Der Quellcode jedes Teammitglieds wird zum Review freigegeben und dabei von mindestens einem anderen Teammitglied kontrolliert. An gemeinsamen Code-Review-Sitzungen werden die Änderungsvorschläge durchgesprochen und in einem Review-Protokoll festgehalten.
\\ Folgende Kriterien werden untersucht:
\\\begin{itemize}
    \item Einhaltung der Code-Richtlinien
    \item Existenz von „Code Smells“
    \item Nachvollziehbarkeit des Codes
\end{itemize}
